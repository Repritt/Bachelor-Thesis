%%%%%%%%%%%%%%%%%%%%%%%%%%%%%%%%%%%%%%%%%%%%%%%%%%%%%%%%%%%%
%% Zacatek vzorove strany %%%%%%%%%%%%%%%%%%%%%%%%%%%%%%%%%%
\thispagestyle{empty}

\noindent {\it Název práce:}\\
{\bf Počítačové modelování reaktoplastů používaných v kotevních sytémech}\\

\noindent
{\it Autor:} Jan Vozáb  \\
\\
\noindent
{\it Obor:}       Konstrukce a dopravní stavby\\
\\
\noindent
{\it Druh práce:} Bakalářská práce\\
\\
\noindent {\it Vedoucí práce:} doc. Ing. Jan Vorel, Ph.D.\\   --------------------------------------------------------------------------------------- \\

\noindent {\it Abstrakt:} 

Reaktoplasty mají v konstrukčním inženýrství důležitou roli. V porovnání s jinými odvětvími, jako je automobilový, letecký a kosmický průmysl, použité reaktoplasty nemusí být vždy v průběhu výstavby plně vytvrzené. Z tohoto důvodu může docházet ke změnám vlastností materiálu v důsledku dodatečného vytvrzování. 

Hlavním cílem této práce je vytvoření numerického modelu, který zachycuje dostatečně přesně vývoj materiálových vlastností a chování reaktoplastů při mechanickém zatěžování. 

Model v této práci je složen ze dvou částí. První je model vytvrzování, který zohledňuje vývoj materiálových parametrů v závislosti na teplotě a času. Jako druhý je použitý elasto-plastický model Drucker-Prager, kterýje využit na popis chování materiálu při mechanickém zatěžování.
\\
\\

\noindent {\it Klíčová slova:} dodatečné vytvrzování, reaktoplasty, Drucker-Prager, metoda konečných prvků

 \newpage
\thispagestyle{empty}
 \noindent
{\it Title:}\\
{\bf Computational modeling of thermoset polymers with application to anchors}\\

\noindent
{\it Autor:} Jan Vozáb \\
 
--------------------------------------------------------------------------------------- \\

\noindent {\it Abstract:} 
Compared to their classical appearance in the aerospace or automotive industry, in civil engineering applications they typically do not reach a fully cured state during construction. Therefore, the material may undergo post-curing causing a significant change in material parameters. The main aim of this work is to create a numerical model that describes sufficiently precisely the evolution of material properties and the behavior of thermosetting polymers during mechanical loading. The model in this thesis is composed of two parts. The first is a curing model that takes into account the development of material parameters in relation to temperature and time. The second is the elasto-plastic model Drucker-Prager, which is used to describe the behavior of the material during mechanical loading.
\\
\\

\noindent 
{\it Key words:} material curing, thermosetting polymers, Drucker-Prager, finite element method
%% Konec vzorove strany %%%%%%%%%%%%%%%%%%%%%%%%%%%%%%%%%%%%
%%%%%%%%%%%%%%%%%%%%%%%%%%%%%%%%%%%%%%%%%%%%%%%%%%%%%%%%%%%%

%\end{document}
