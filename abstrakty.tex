%%%%%%%%%%%%%%%%%%%%%%%%%%%%%%%%%%%%%%%%%%%%%%%%%%%%%%%%%%%%
%% Zacatek vzorove strany %%%%%%%%%%%%%%%%%%%%%%%%%%%%%%%%%%
\thispagestyle{empty}

\noindent {\it N\'azev pr\'ace:}\\
{\bf Fluktuace multiplicity ve srážkách těžkých iontů}\\

\noindent
{\it Autor:} Bc. Radka Sochorov\'a  \\
\\
\noindent
{\it Obor:}       Experiment\'aln\'i  jadern\'a  a \v{c}\'asticov\'a  fyzika\\
\\
\noindent
{\it Druh práce:} Diplomová práce\\
\\
\noindent {\it Vedoucí práce:} Doc. Dr. Boris Tom\'a\v{s}ik\\  --------------------------------------------------------------------------------------- \\

\noindent {\it Abstrakt:} 

V rané fázi ultrarelativistických srážek těžkých iontů je produkována horká a hustá jaderná hmota, kterou nazýváme kvark-gluonovým plazmatem (QGP). Jde o nedávno objevené médium, ve kterém už běžné hadrony neexistují a ve kterém se kvarky a gluony stávají volnými. 

Hlavní motivací této práce je pozorování, že celkové pozorované multiplicity jednotlivých druhů částic z ultrarelativistických srážek těžkých iontů souhlasí se statistickým modelem při teplotách nad $160 ~\mathrm{MeV}$. Teplotu fázového přechodu lze stanovit také pomocí metod QCD na mřížce. Je to teplota, při které se susceptibility různých řádů jako funkce teploty mění nejrychleji. Tato teplota se nachází okolo $150 ~\mathrm{MeV}$. Susceptibility různých řádů se přitom projevují ve vyšších momentech rozdělení multiplicity.

Cílem této práce je studovat vývoj rozdělení multiplicity za pomoci řídící rovnice. Nejdříve se zaměřujeme na vyšší faktoriální momenty (třetí a čtvrtý), ze kterých můžeme následně odvodit všechny ostatní momenty jako např. centrální momenty. Z nich pak můžeme spočítat jejich důležité kombinace, koeficient šikmosti a koeficient špičatosti. Jelikož nás zajímá čas termalizace faktoriálních momentů, nejdříve studujeme vývoj rozdělení multiplicity pro konstantní teplotu. Potom necháme teplotu fireballu postupně klesat a sledujeme, jak se vyšší momenty chovají v tomto případě.  
\\
\\
\noindent {\it Klíčová slova:} fázový přechod, řídící rovnice, faktoriální momenty, centrální momenty, koeficient šikmosti a špičatosti

 \newpage
\thispagestyle{empty}
 \noindent
{\it Title:}\\
{\bf Multiplicity fluctuations in heavy-ion collisions}\\

\noindent
{\it Autor:} Bc. Radka Sochorová \\
--------------------------------------------------------------------------------------- \\

\noindent {\it Abstract:} 

In the early phase of an ultrarelativistic heavy ion collisions is produced hot and dense nuclear matter, which is called the quark-gluon plasma (QGP). It is a newly-discovered form of matter, in which ordinary hadrons do not exist anymore, and in which quarks and gluons become free. 

The main motivation of this work is that overall observed multiplicity of different types of particles from ultrarelativistic nuclear collisions agrees with the statistical model at temperatures above $160 ~\mathrm{MeV}$. The phase transition temperature can be also determined by lattice QCD methods. It is a temperature at which susceptibilities of different orders as functions of temperature are changing fastest. This temperature is about $150 ~\mathrm{MeV}$. Susceptibilities of different orders manifest
themselves in higher moments of the multiplicity distribution.

So the main aim of this work is to study the evolution of the multiplicity distribution with the help of a master equation. Particularly we focus on the higher factorial moments (third and fourth) from which all other kinds of moments, eg. central moments, can be calculated. From the central moments we can derive their important combinations, coefficient of skewness and kurtosis. We first study thermalisation time of factorial moments when the temperature is fixed. Then we study the evolution of the moments 
in a situation with decreasing temperature. 
\\
\\
\noindent {\it Key words:} phase transition, Master equation,  factorial moments, central moments, coefficient of skewness and kurtosis
%% Konec vzorove strany %%%%%%%%%%%%%%%%%%%%%%%%%%%%%%%%%%%%
%%%%%%%%%%%%%%%%%%%%%%%%%%%%%%%%%%%%%%%%%%%%%%%%%%%%%%%%%%%%

%\end{document}
