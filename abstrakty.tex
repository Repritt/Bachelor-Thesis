%%%%%%%%%%%%%%%%%%%%%%%%%%%%%%%%%%%%%%%%%%%%%%%%%%%%%%%%%%%%
%% Zacatek vzorove strany %%%%%%%%%%%%%%%%%%%%%%%%%%%%%%%%%%
\thispagestyle{empty}

\noindent {\it Název práce:}\\
{\bf Počítačové modelování reaktoplastů používaných v kotevních sytémech}\\

\noindent
{\it Autor:} Jan Vozáb  \\
\\
\noindent
{\it Obor:}       Konstrukce a dopravní stavby\\
\\
\noindent
{\it Druh práce:} Bakalářská práce\\
\\
\noindent {\it Vedoucí práce:} doc. Ing. Jan Vorel, Ph.D.\\   --------------------------------------------------------------------------------------- \\

\noindent {\it Abstrakt:} 

Thermosetové polymery mají v konstrukčním inženýrství důležitou roli. Avšak hlavní rozdíl od použití v automobilovém nebo leteckém inženýrství spočívá v tom, že při použití v konstruknční fázi výstavby není polymer obvykle dokonale vytvrzený, v důsledku čehož se stává komplikovanější simulace tohoto materiálu.

Hlavní motivací této práce pozorování chování thermosetových polymerů a popsání jejich vlastností s ohledem na dodatečné vytvrzování a s tím spojenou změnu materiálových vlastností. 

Cílem této práce je popsání a implementace materiálového modelu, který zohledňuje výše popsané vlivy a je schopen simulovat chování materiálu zatíženým v rešiči na bázi metody konečných prvků. Model v této práci je složen ze dvou částí. První je model vytvrzování, který zohledňuje vytvrzování materiálu v závislosti na teplotě a času, a jehož výstupem jsou upravené materiálové parametry modelu. Jako druhý je použitý elasto-plastický model Drucker-Prager, který popisuje materiál s ohledem na mez plasticity a plastického chování. 
\\
\\

\noindent {\it Klíčová slova:} dodatečné vytvrzování, Viskoelastický materiál, thermosetové polymery, Drucker-Prager, metoda konečných prvků

 \newpage
\thispagestyle{empty}
 \noindent
{\it Title:}\\
{\bf Computational modeling of thermoset polymers with application to anchors}\\

\noindent
{\it Autor:} Jan Vozáb \\
 
--------------------------------------------------------------------------------------- \\

\noindent {\it Abstract:} 
Thermosetting polymers have an important role in structural engineering. However main difference between using this material in automotive and aerospace engineering is in fact, that in this region is used often in fully cured state, which leads to difficult simulation of this material. 

The main motivation of this work is to observe the behavior of thermoset polymers and to describe their properties with regard to the additional curing and the associated change in material properties.

So the aim of this work is to describe and implement a material model that takes into account the above described effects and is able to simulate the behavior of the material loaded in the finite element method. The model in this thesis is composed of two parts. The first is a curing model that takes into account curing of material  in relation to temperature and time, and the outcome are the adjusted material parameters. The second is the elasto-plastic model Drucker-Prager, which describes the material with respect to the limits of plasticity and plastic behavior.
\\
\\

\noindent 
{\it Key words:} material curing, viscoelastic material,  thermosetting polymers, Drucker-Prager, finite element method
%% Konec vzorove strany %%%%%%%%%%%%%%%%%%%%%%%%%%%%%%%%%%%%
%%%%%%%%%%%%%%%%%%%%%%%%%%%%%%%%%%%%%%%%%%%%%%%%%%%%%%%%%%%%

%\end{document}
