
\mbox{}
\thispagestyle{empty}
\clearpage

\section{Introduction}

In the last century construction engineering as well aerospace engineering  were dominated by the materials steel, aluminum, and concrete. But especially in last decade civil engineers more than ever faced often contradictory demands for designing larger, safer and more durable structures at shorter time and lower costs. This lead to improvement of old as designing new materials. Composites are a key element of those new designs. \todo[size=\tiny]{pokus}

Composite materials often combine positive characteristic properties from more, typically two different materials which result to better material properties. In many cases these combine a load carrying constituent typically in the form of carbon or glass fibers bonded to the cement or polymer based matrices. Their applications can be found in transportation as well in civil engineering fields. In the aerospace industry we can found entire structural members made from composite materials, but in the building industry is using of polymer-based composites limited. A typical area are members applied to existing concrete or masonry such as adhesive anchors. The commonly used polymers face challenges as:

\begin{itemize}
	\item typically utilized exothermically reacting thermosetting polymers, e.g. epoxies or vinyl-esters, with defined post-curing,
	\item high filler content, including even cement and water,
	\item uncertain curing level and the mechanical properties depending on the environmental conditions (a fully cured state is not usually reached).
\end{itemize} 

Also a large range of working temperatures, which are typically expected during the lifetime leads to post-curing and related changes in mechanical properties. These changes highly impact in particular structures under sustained or cyclic loads. For these challenges, the characterization of this type of materials is in high demand.  