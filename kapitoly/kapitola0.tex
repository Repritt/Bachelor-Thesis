\section*{Introduction}
\addcontentsline{toc}{section}{\protect\numberline{}Introduction}
\indent

In the last century construction engineering as well aerospace engineering  were dominated by the materials steel, aluminum, and concrete. But especially in last decade civil engineers more than ever faced often contradictory demands for designing larger, safer and more durable structures at shorter time and lower costs. This lead to improvement of old and designing of new materials. Composites are a key element of those new designs.

Composite materials often combine positive characteristic properties from more, typically two, different materials which result to better material properties. In many cases these combine a load carrying constituent, typically in the form of carbon or glass fibers, bonded to the cement or polymer based matrices. Their applications can be found in transportation as well as in civil engineering fields. In the aerospace industry we can found entire structural members made of composite materials, but in the building industry we can find the use of polymer-based composites is limited. A typical area are members applied to existing concrete or masonry such as adhesive anchors. The commonly used polymers are typically utilized, exothermically reacting, thermosets, e.g. epoxies or vinyl-esters. They have high filler content (including even cement and water). They have uncertain curing level and the mechanical properties due to the environmental conditions (a fully cured state is not usually reached).

Also a large range of working temperatures, which are typically expected during the lifetime leads to a post-curing and related changes in mechanical properties. These changes highly impact in particular structures under sustained or cyclic loads. For these challenges, the characterization of this type of materials is in high demand.  

The first chapter is focused on the description of anchors, the distribution according to the installation time and the load transfer mechanism, further are here described failure models of anchors.

The second chapter is intended to describe thermosets, used types of polymers with anchors and overall material properties. Also there are explained levels of complexity through which can be polymers simulated. 

In third chapter are explained models, which are used for numerical simulation of thermosetting polymers. As first of them is explained non-associated Drucker-Prager model used for calculation of current stress and strain in time. The second model described is the curing model, which is employed to calculate the evolution of material parameters.

The fourth chapter is focused on the results and their comparison with experimental test data.

