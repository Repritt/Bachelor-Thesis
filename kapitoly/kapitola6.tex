\thispagestyle{plain}
\section*{Conclusion and future work}
\addcontentsline{toc}{section}{\protect\numberline{}Conclusion}
\indent

The main aim of this work was to create a numerical model that describes sufficiently precisely the evolution of material properties and the behavior of thermoset polymers during the mechanical loading. In the first chapter, the different types of anchors are described and the divisions based on the installation time and the load transfer mechanism are presented. Finally, the failure modes of anchors are briefly summarized. The thermosetting polymers and their types are characterized in Chapter 2. The utilized numerical models together with the implementation procedure are described in Chapter 3. First the Drucker-Prager model is studied and its implementation into the FE software is presented. Then the curing model for polymers is briefly described and the evolution of material parameters is based on the curing degree. The idea of serially coupled models is also defined in Chapter 3. Finally, the results supporting the use of studied models are presented in Chapter 4. It should be noted that only preliminary results showing the capability of individual models are presented and more complex study is needed to really verify the aforementioned models. Moreover, some extensions of the models as suggested in [9] and presented results may be needed. 

In general, the development of complex material model is a crutial step towards the successful modelling of bonded anchors. The presented work is the first attempt to fulfill this goal. As shown, the standard Drucker-Prager model may need some additional improvements and therefore the utilization of microplane material model \cite{di2007symmetric} is expected in the future. The free volume approach presented in \cite{popelar1997multiaxial} will be also employed to simulate the time, temperature and humidity influence. These additional improvements and extensions are expected to be a part of the Master’s thesis.
