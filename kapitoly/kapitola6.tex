\thispagestyle{plain}
\section*{Conclusion}
\addcontentsline{toc}{section}{\protect\numberline{}Conclusion}
\indent

The main aim of this work was to create a numerical model that describes sufficiently precisely the evolution of material properties and the behavior of thermosetting polymers during the mechanical loading.

In the first chapter was described anchors. There were shown that they can be divided by two main conditions: installation time and load transfer mechanism. By the installation time they can be splitted into the cast-in-place and the post-installed, and from the point of view of load transfer mechanism we found three main transfer principles, specifically friction, keying and bonding. As the next we divided the types of post-installed anchors, which was expansion, undercut, screw and adhesive. Finally, we discussed in this chapter the failure models of anchors and described their dependence on loading and load transfer mechanism. The second chapter is focused on describing of the thermosetting polymers, their types used in mortars and possible numerical description, which can be used. Then we described material properties, which can affect visco-elastic behavior of the material. In third chapter are described models used for computational modeling of the thermosetting polymers in this thesis, first is Drucker-Prager model of plasticity and its implementation. Second is Curing model for the simulation of curing effects and theirs influence to the material parameters. Then is described idea of serially connected models, or possibility of different approach with implementing more complex model solution. In the fourth chapter are presented results of implemented models. After comparing results from Drucker-Prager model with experimental data, the presented model does not describe the data precisely. And there is the possibility of improvement with more complex cohesion and friction chain or implementing Cap model into Drucker-Prager yield equation. Last but not least there were shown results from the Curing model. This model was already implemented into the Mars solver. It shown results compared with the presented article, where was model introduced and shown that it is working properly. 

For more precise simulation of this type of material is necessary to improve the model, which describe mechanical behavior. That could be done with addition of the Cap model to the solver. Or the complete Drucker-Prager could be changed by different model. Then is necessary to connect models for curing and mechanical behavior together, because we want to simulate visco-elastic behavior. If the results will not be accurate enough, then the approach of implementing more complex model Microplane M4 could be done. This could be, we hope, part of the master's thesis. 